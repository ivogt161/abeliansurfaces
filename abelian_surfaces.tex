\documentclass[11pt]{amsart}
\usepackage{latexsym}
\usepackage{amssymb,amsmath,mathabx, amscd}
\usepackage[pdftex]{graphicx}
\usepackage{enumerate}
\usepackage{tikz}
\usetikzlibrary{snakes}
\usepackage[margin=1in]{geometry}
%\hypersetup{colorlinks=true,urlcolor=RawSienna,citecolor=RoyalPurple,linkcolor=MidnightBlue}
\usepackage{listings}
\usepackage{courier}
\usepackage{tikz-cd}
\usepackage{color}
\lstset{
	basicstyle=\small\ttfamily,
	keywordstyle=\color{blue},
	language=python,
	xleftmargin=16pt,
}
\usepackage{MnSymbol}
\usepackage{hyperref}
\usepackage{mathrsfs, colonequals}

\newcommand{\aff}[1]{\mathbb{A}_k^{#1}}

\renewcommand{\aa}{\mathbb{A}}
\newcommand{\cc}{\mathbb{C}}
\newcommand{\rr}{\mathbb{R}}
\newcommand{\pp}{\mathbb{P}}
\newcommand{\hh}{\mathbb{H}}
\newcommand{\qq}{\mathbb{Q}}
\newcommand{\zz}{\mathbb{Z}}
\newcommand{\ff}{\mathbb{F}}
\newcommand{\kk}{\mathbb{K}}
\renewcommand{\gg}{\mathbb{G}}
\newcommand{\nn}{\mathbb{N}}
\renewcommand{\tt}{\mathbb{T}}

\newcommand{\U}{\mathcal{U}}
\newcommand{\I}{\mathcal{I}}
\renewcommand{\H}{\mathcal{H}}
\newcommand{\fH}{\mathscr{H}}
\renewcommand{\O}{\mathcal{O}}
\newcommand{\E}{\mathcal{E}}
\newcommand{\F}{\mathcal{F}}
\newcommand{\G}{\mathcal{G}}
\renewcommand{\P}{\mathcal{P}}
\renewcommand{\S}{\mathcal{S}}
\newcommand{\Q}{\mathcal{Q}}
\newcommand{\T}{\mathcal{T}}
\renewcommand{\L}{\mathcal{L}}
\newcommand{\M}{\mathcal{M}}

\newcommand{\sD}{\mathscr{D}}
\newcommand{\sE}{\mathscr{E}}
\newcommand{\sL}{\mathscr{L}}
\newcommand{\sQ}{\mathscr{Q}}

\newcommand{\p}{\mathfrak{p}}
\newcommand{\m}{\mathfrak{m}}

\newcommand{\Res}{\text{Res}}
\newcommand{\ord}{\text{ord}}
\newcommand{\GL}{\text{GL}}
\newcommand{\SL}{\text{SL}}
\newcommand{\PSL}{\text{PSL}}
\newcommand{\GSp}{\text{GSp}}
\newcommand{\SO}{\text{SO}}
\newcommand{\Sp}{\text{Sp}}
\renewcommand{\sl}{\mathfrak{sl}}
\newcommand{\PGL}{\text{PGL}}
\newcommand{\Sym}{\operatorname{Sym}}
\newcommand{\Supp}{\operatorname{Supp}}
\newcommand{\Pic}{\operatorname{Pic}}
\newcommand{\Jac}{\operatorname{Jac}}
\newcommand{\res}{\operatorname{res}}
\newcommand{\sm}{\operatorname{sm}}


%\newcommand{\Id}{\operatorname{Id}}
%\newcommand{\rk}{\operatorname{rk}}
%\newcommand{\td}{\operatorname{td}}
%\newcommand{\ch}{\operatorname{ch}}
\newcommand{\Hilb}{\operatorname{Hilb}}
%\newcommand{\Coh}{\operatorname{Coh}}
%\newcommand{\Pic}{\operatorname{Pic}}
\newcommand{\NS}{\operatorname{NS}}
\newcommand{\Bl}{\operatorname{Bl}}


\newcommand{\bound}{\partial}
\renewcommand{\Im}{\operatorname{Im}}
\newcommand{\Ast}{\Asterisk}
\newcommand{\tr}{\operatorname{tr}}
\newcommand{\sgn}{\operatorname{sgn}}
\newcommand{\Nm}{\operatorname{Nm}}
\newcommand{\Gal}{\operatorname{Gal}}
\newcommand{\Frob}{\operatorname{Frob}}
\newcommand{\Mat}{\operatorname{Mat}}
\newcommand{\Aut}{\operatorname{Aut}}
\newcommand{\Ann}{\operatorname{Ann}}
\newcommand{\rad}{\operatorname{rad}}
\newcommand{\Hom}{\operatorname{Hom}}
\newcommand{\Ram}{\operatorname{Ram}}
\newcommand{\rk}{\operatorname{rk}}
\newcommand{\ev}{\operatorname{ev}}
\newcommand{\gen}{\operatorname{gen}}

\newcommand{\cyc}{\operatorname{cyc}}
\renewcommand{\top}{{\text{top}}}

\newcommand{\del}{\partial}

\renewcommand{\bar}{\overline}

\newcommand{\fixme}[1]{\noindent{\color{red}FIXME: #1 }}

\newcommand{\isabel}[1]{{\color{purple} ($\spadesuit$ Isabel: #1)}}
\newcommand{\padma}[1]{{\color{teal} ($\clubsuit$ Padma: #1)}}


\newtheorem{thm}{Theorem}[section]
\newtheorem{statement}[thm]{Statement}
\newtheorem{ithm}{Theorem}
\newtheorem{icor}[ithm]{Corollary}
\newtheorem{lem}[thm]{Lemma}
\newtheorem{conj}[thm]{Conjecture}
\newtheorem{prop}[thm]{Proposition}
\newtheorem{cor}[thm]{Corollary}
\newtheorem{cond}[thm]{Condition}
\newtheorem{claim}[thm]{Claim}

\newcommand{\defi}[1]{\textsf{#1}} 

\theoremstyle{definition}
\newtheorem{defin}[thm]{Definition}
\newtheorem{example}[thm]{Example}
\newtheorem{exercise}[thm]{Exercise}
\newtheorem*{question}{Main Question}

\theoremstyle{remark}
\newtheorem{rem}{Remark}
\newtheorem*{skproof}{Sketch of Proof}


%\renewcommand\footnotemark{}
%\renewcommand\footnoterule{}

\title{Plan for ICERM project group}
\author{Padmavathi Srinivasan}
\author{Isabel Vogt}

\date{\today}


\begin{document}

\maketitle

The purpose of this document is to outline an approach to the problem for the ICERM working group: given a principally polarized abelian surface $A$ over $\qq$ with endomorphism ring $\zz$, return the (provably complete) finite list of primes $\{\ell_1, \dots, \ell_n\}$ for which the Galois representation on the $\ell_i$-torsion points
\[\rho_{A, \ell_i} \colon G_{\qq} \to \Aut\left(A[\ell_i]\right) = \GSp_{4}(\ff_{\ell_i}) \]
is not surjective (where we write $G_{\qq}$ for $\Gal\left(\bar{\qq}/\qq\right)$).  
%(More generally we might want to find all \emph{non-surjective} primes where $\rho_{A, \ell}(G_{\qq})$ is a proper subgroup of $\GSp_{4}(\ff_{\ell_i})$.  If we want, we can think about implementing this after the current project is completed.  The problem of determining reducible primes is the first step in this direction.)

The overarching idea is to test Frobenius elements at primes $p$ of good reduction for $A$.  We can determine the (integral) characteristic polynomial of Frobenius at $p$ by counting points on $A$ over $\ff_{p^r}$ for small $r$.    The reduction of this polynomial modulo $\ell$ gives the characteristic polynomial of the action of the Frobenius element on the $\ell$-torsion points of $A$.  By the Chebotarev density theorem, the images of the Frobenius Galois elements for varying primes $p$ equidistribute over the conjugacy classes of $\rho_{A, \ell}(G_{\qq})$.  In this way, we can sample the elements of this group.

This problem naturally splits into two parts:
\begin{enumerate}
\item Find a finite list of primes that provably contains all primes for which the mod $\ell$ representation could be non-surjective.
\item Given a prime $\ell$, determine if the mod $\ell$ representation $\rho_{A, \ell}$ is not surjective.
\end{enumerate}
We will treat these cases separately in two sections.  We begin by reviewing the properties of the characteristic polynomial of Frobenius with a view towards computational aspects.


\section{Characteristic polynomials of Frobenius}

Let $A$ be an abelian variety of dimension $g$ defined over $\qq$.
We will write $T_\ell A$ for the $\ell$-adic Tate module of $A$:
\[T_\ell A \simeq \varprojlim_n A[\ell^n].\]
This is a free $\zz_\ell$-module of rank $2g$.


Let $p$ be a prime of good reduction for $A$.  We will write $\Frob_p$ for the image of a Frobenius element at $p$ in $G_\qq$ acting on $T_\ell A$.  (This is well-defined up to conjugacy.)  The theoretical result underlying this entire approach is the following.

\begin{thm}[Deligne (is there a better ref?)]\label{integrality}
Let $A$ be an abelian variety defined over $\qq$ and let $p$ be a prime of good reduction for $A$.  There exists a monic integral polynomial $P_p(t) \in \zz[t]$ of degree $2g$ with constant coefficient $p^g$ such that for any $\ell \neq p$, $P_p(t)$ is the characteristic polynomial of the action of $\Frob_p$ on $T_\ell A$.
\end{thm}

There is further structure.  The nondegenerate Weil pairing gives an isomorphism (of Galois modules):
\[T_\ell A \simeq \left(T_\ell A\right)^\vee \otimes_{\zz_\ell} \zz_\ell(1).\]
The Galois character acting on $\zz_\ell(1)$ is the \defi{$\ell$-adic cyclotomic character}, which we denote by $\cyc_\ell$.  
The integral characteristic polynomial for the action of $\Frob_p$ on $\zz_\ell(1)$ is simply $t-p$.  The integral characteristic polynomial for the action of $\Frob_p$ on $\left(T_\ell A\right)^\vee$ is the reversed polynomial
\[P^\vee_p(t) = P_p(1/t)\cdot t^{2g}/p^g \]
whose roots are the inverses of the roots of $P_p(t)$.

The structure of the nondegenerate Weil pairing gives that the (complex) roots of $P_p(t)$ come in pairs
\[\alpha, \frac{p}{\alpha} \]
that multiply to $p$.

Let us now specialize to the case that $A$ has dimension $2$.
Suppose that $\alpha$, $\beta$, $p/\alpha$, and $p/\beta$ are the complex roots of the $P_p(t)$.  By Theorem \ref{integrality}, $P_p(t) \mod{\ell}$ is the characteristic polynomial of the action of $\Frob_p$ on $A[\ell]$.  We therefore have the following necessary condition:

\begin{cond}\label{easy_necessary}
If $A[\ell]$ is reducible, then $P_p(t)$ is not irreducible modulo $\ell$.
\end{cond}

Our goal in Section \ref{finite_list} will be to deduce more useful necessary conditions that will allow us to reduce to a finite list of possible $\ell$.  Our goal in Section \ref{test_ell} will, essentially, be to use Condition \ref{easy_necessary} to test if $A[\ell]$ is indeed reducible.  In that section we must understand when this condition fails to be sufficient (i.e., solve a local-global problem).


The polynomials $P_p(t)$ are computationally accessible by counting points on $A$ over $\ff_{p^r}$ for small $r$.  \isabel{This should be expanded some.}  
By the Grothendieck--Lefschetz trace formula (Section 5.4 of curves.pdf), if $A= \Jac(X)$, and if $\lambda_1,\lambda_2,\ldots,\lambda_{2g}$ are the roots of $P_p(t)$, then
\[ \# X(\ff_{p^r}) = p^r+1-\sum_{i=1}^{2g} \lambda_i^r. \]

\padma{Sage Implementation: \\

https://doc.sagemath.org/html/en/reference/curves/sage/schemes/hyperelliptic\_curves/hyperelliptic\_finite\_field.html \\

http://www-math.mit.edu/~poonen/papers/curves.pdf }
% Sections 3.7 and 4.2.2 might also help

\section{Finding a finite set containing reducible primes}\label{finite_list}

To set some notation, suppose that $\rho_{A, \ell}$ is reducible.  In other words, the representation $A[\ell]$ is an extension
\begin{equation}\label{red_extn} 0 \to X_\ell \to A[\ell] \to Y_\ell \to 0\end{equation}
of the (quotient) representation $Y_\ell$ by the (sub) representation $X_\ell$.
In other words, in an appropriate basis, the image of $\rho_{A, \ell}$ is block diagonal:

\begin{center}
\begin{tikzpicture}[scale=.9]

\draw[thick] (-2, 2) -- (-2, -2);
\draw[thick] (2, 2) -- (2, -2);

\draw (-1, 2) -- (-1, -2);
\draw (-2, 1) -- (2, 1);


\node at (.5, -.5) {$Y_{\ell}$};

\draw[snake=brace, raise snake=5pt] (-2, 2) -- (-1, 2);
\node at (-1.5, 1.5) {$X_{\ell}$};
\node at (-1.5, 2.5) {$x$};

\draw[snake=brace, mirror snake, raise snake=5pt] (-1, -2) -- (2, -2);
\node at (.5, -.5) {$Y_{\ell}$};
\node at (.5, -2.5) {$y$};

\end{tikzpicture}
\end{center}

We will write $x$ for the dimension of the subrepresentation $X_\ell$ and $y$ for the dimension of the quotient representation $Y_\ell$.

The key result that will allow us to constrain the possible reducible $\ell$ is the following Lemma (see Serre for the case of elliptic curves):

\begin{lem}
Suppose that $\ell$ is a prime of semistable reduction for $A$.  Let $\cyc_\ell$ be the mod $\ell$-cyclotomic character.
In the above setup, there exist integers $0 \leq u \leq 120x$ and $0 \leq v \leq 120 y$, both divisible by $120$, such that $u + v = 240$ and
\begin{align*}
\left( \det X_{\ell} \right)^{120} &= \left(\cyc_\ell\right)^u, \\
\left( \det Y_{\ell} \right)^{120} &= \left(\cyc_\ell\right)^v.
\end{align*}
\end{lem}
\begin{skproof}
\isabel{We can blackbox this.  I can also find references later.} This follows from the following two results:
\begin{enumerate}
\item (Grothendieck) An abelian variety $A$ has semistable reduction at $p \neq \ell$ if and only if the action of inertia at $p$ on $T_\ell A$ is unipotent.
\item (Raynaud) Given a finite flat group scheme killed by $\ell$ over $\qq_\ell$, determines the possibilities for the action of inertia at $\ell$ on the determinant.  Basically, the only possibility is a power of the cyclotomic character (where the power is at most the dimension).
\end{enumerate}
Since there are no unramified characters of $G_\qq$, it suffices to understand the determinant characters on the inertia groups $I_p$ for all primes $p$.

Every abelian surface obtains semistable reduction over an extension of degree at most $120$.  Therefore, the $120$th power of every inertial element for $p\neq \ell$ acts trivially on 
\qed
\end{skproof}

In order to use this Lemma.  There are two cases to consider:
\begin{enumerate}
\item\label{odd} One of $x$ or $y$ is $1$, or
\item\label{even} Both $x$ and $y$ are $2$.
\end{enumerate}

In case \eqref{odd}, let us assume that $x=1$.  Then for any good prime $p$, the integral characteristic polynomial of Frobenius at $p$ splits off a linear factor modulo $\ell$ (corresponding to the action on the subrepresentation $X_\ell$).  Furthermore, by the Lemma, either
\[X_{\ell}^{120} = 1, \qquad \text{or} \qquad X_{\ell}^{120} = \left(\cyc_\ell\right)^{120}.\]
We will use the fact that we know how the characteristic polynomial of Frobenius at $p$ acts on the $1$-dimensional $\ff_\ell$-representation whose Galois character is either $1$ or $\cyc_\ell^{120}$; namely, by either $1$ or multiplication by $p^{120}$ (modulo $\ell$).
Hence, for any good prime $p$, the characteristic polynomial of Frobenius at $p$ has a root whose $120$th power is either $1$ or $p^{120}$ modulo $\ell$.
Given the (integral) characteristic polynomial $P_p(t)$ for Frobenius at $p$, using linear algebra of symmetric functions, we can write down an integral polynomial $P_p^{(120)}(t)$ whose roots are the $120$th powers of the roots of $P_p(t)$.  By the Weil conjectures, the roots of $P_p(t)$ have absolute values $p^{1/2}$; hence the roots of $P^{(120)}(t)$ have absolute value $p^{60}$.  Therefore both $P_p^{(120)}(1)$ and $P_p^{(120)}(p^{120})$ are \emph{non-zero} integers.  And if we are in this case ($A[\ell]$ is reducible with either $1$-dimensional sub or quotient) then $\ell$ divides the nonzero quantity $P_p^{(120)}(1)\cdot P_p^{(120)}(p^{120})$.

Now let us assume that $A[\ell]$ does not have a subrepresentation of odd dimension (and hence we are in case \eqref{even}).  This is more subtle, as we now explain.  We want to use the fact the $120$th power of the product of the two eigenvalues of Frobenius at $p$ acting on $X_\ell$ (or $Y_\ell$) multiply to $1$, $p^{120}$ or $p^{240}$ modulo $\ell$.  The issue is that the nondegeneracy of the Weil pairing guarantees the existence of two pairs of roots of $P_p(t)$ that \emph{always} multiply to $p$ (and hence the $120$th power of their product is $p^{120}$ \emph{integrally}).  We therefore need more care in this case.

By the nondegeneracy of the Weil pairing, we have that
\begin{equation}\label{weil_pair}A[\ell] \simeq A[\ell]^\vee \otimes \cyc_\ell.\end{equation}
If $A[\ell]$ is reducible but indecomposable (e.g., equation \eqref{red_extn} is non-split), then $X_{\ell}$ is the unique subrepresentation of $A[\ell]$ and $Y_\ell^\vee \otimes \cyc_{\ell}$ is the unique subrepresentation of $A[\ell]^\vee \otimes \cyc_\ell$.  Hence
\[ X_\ell \simeq Y_\ell^\vee \otimes \cyc_\ell.\]
Otherwise $A[\ell] \simeq X_\ell \oplus Y_\ell$ and so the nondegeneracy of the Weil pairing \eqref{weil_pair} gives
\[X_\ell \oplus Y_\ell \simeq \left(X_\ell^\vee \otimes \cyc_\ell\right) \oplus \left(Y_\ell^\vee \otimes \cyc_\ell\right).\]
Therefore either:
\begin{enumerate}[(a)]
\item $X_\ell \simeq Y_\ell^\vee \otimes \cyc_\ell$ and $Y_\ell\simeq X_\ell^\vee \otimes \cyc_\ell$ (the only possibility if $A[\ell]$ is indecomposable), or 
\item $X_\ell \simeq X_\ell^\vee \otimes \cyc_\ell$ and $Y_\ell \simeq Y_\ell^\vee \otimes \cyc_\ell$.
\end{enumerate}

Suppose that $\alpha$, $\beta$, $p/\alpha$, and $p/\beta$ are the complex roots of the $P_p(t)$.  By counting points we will be able to compute
\begin{align*}
t &\colonequals \alpha + \beta + \frac{p}{\alpha} + \frac{p}{\beta} \\
s &\colonequals \alpha^2 + \beta^2 + \frac{p^2}{\alpha^2} + \frac{p^2}{\beta^2} 
\end{align*}

In case (a) above, $\alpha$ and $\beta$ are the roots of the action of Frobenius at $p$ on $X_\ell$.  We can then form the degree $4$ polynomial whose roots are the products of the pairs of roots that don'
t automatically multiply to $p$ can be expanded to be:
\[Q_p(t) \colonequals x^4 - \left( \frac{t^2 - s - 4p}{2} \right) x^3 + (2p^2 + ps)x^2 - p^2\left( \frac{t^2 - s - 4p}{2} \right)x + p^4.\]
We can now repeat the previous idea with $Q_p^{(120)}(t)$ evaluated on $1$, $p^{120}$ and $p^{240}$.

In case (b), this will not work since $\alpha$ and $p/\alpha$ are the roots expressing the action on $X_\ell$.  Write 
\[F_p(t) \colonequals (t - (\alpha + p/\alpha))(t - (\beta + p/\beta)).\]
We can now use the modularity theorem (Serre's conjecture).  This says that $X_\ell$ is the mod $\ell$ Galois representation arising from a cusp form of weight $2$ and level dividing $\ell$ times the conductor of the abelian surface.  Let $\Sigma$ be the set of all possible cusp forms $f$.  Write $a_p(f)$ for the eigenvalue of the Hecke operator $T_p$ for $f$.  Therefore
\[\ell \mid \prod_{f \in \Sigma} F_p(a_p(f)).\]
This again gives us a nontrivial condition on $\ell$.

We turn this into an algorithm to determine $\ell$ as follows:
\begin{enumerate}
\item
\item
\end{enumerate}


\section{Determining if a representation is reducible}\label{test_ell}






\end{document}
